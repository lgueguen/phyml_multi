\documentclass[a4paper,11pt]{article}
\usepackage[dvips]{epsfig}
\usepackage{floatflt}
\usepackage{fleqn}
\usepackage{amssymb,amstext}
\usepackage{array}
\usepackage[latin1]{inputenc}
\usepackage{newicktree}
\usepackage{graphicx}
\usepackage{amsmath}
\usepackage[margin=1in]{geometry}
\usepackage{array}
\usepackage{amsmath}
\usepackage{latexsym}
\usepackage{psfrag}
\usepackage{graphicx}
\usepackage{tablo}
\usepackage{ae,aecompl,aeguill}
\usepackage{setspace}
\usepackage{helvetic}
\usepackage{caption2}
\usepackage{url}
\usepackage{cite}          
\usepackage{chicago}
\usepackage{wasysym}

\urlstyle{tt}


\renewcommand\captionmargin{0cm} 
\renewcommand{\captionlabelfont}{\bf}
\renewcommand{\captionlabeldelim}{}
%% \captionstyle{centerlast}


\renewcommand\citeleft{(}
\renewcommand\citeright{)}
\renewcommand\citepunct{; }


\newcommand{\rep}[3][1]
{
\psfrag{#2}[c][c][#1]{#3}
}

\newcommand{\bm}[1]
{
\mbox{\boldmath$#1$}
}

\renewcommand{\baselinestretch}{1.8}


\begin{document}

\title{PHYML technical documentation}
\maketitle
Substitution models in PHYML correspond to homogeneous, stationary and
time-reversible Markov  processes. Therefore, the  likelihood does not
depend on the position of the root  of the phylogeny. Let $r$ be this root,
and $R$ denotes  the tree.  $u$ and  $v$ are the roots of subtrees $U$
and $V$ respectively.  $u$ is a tip of $R$ that  is separated from $v$
by a single branch of length $l$. $\pi_h$ is the equilibrium frequency
of state  $h$.  $P_{hh^{\prime}}(l)$ is the probability  for the state
$h$  to  be replaced  by  state  $h^{\prime}$  after $l$  substitution
events.   The substitution  rate  at  each site  is  distributed as  a
discretized  gamma distribution  with  $G$ categories.  $r_g$ is  the
relative    rate    for   cateogory    $g$    and    $p_g$   is    its
probability. $\mathcal{A}$  is the state  space to be  considered.  We
define  the conditional  likelihood $L(s=h|U)$  as the  probability of
data at site $s$ given that node $u$ has state $h$. $L(s=h|V)$ has the
same meaning when  $V$ (and $v$) replaces $U$ (and  $u$). $L^{*}(s=h|V) =
L(s=h|V) / Z_s(V)$ and $L^{*}(s=h|U) = L(s=h|U) / Z_s(U)$ are scaled
conditional likelihoods. $Z_s(V)$ and $Z_s(U)$ are  scale factors that
are used to avoid numerical underflows when computing very small values of
conditional likelihoods. We first consider
the case where the state observed at the leaf $u$ is not ambiguous. The
scaled-likelihood at site $s$ is then~:
\begin{equation}
L^{*}(s) = 
\sum_{g}^{G}\sum_{h \in \mathcal{A}}
p_{g}
\pi_h
L^{*}(s=h|V)
P_{hh^{\prime}}(l \times r_g)
\end{equation}

If the state at tip $u$ is ambiguous, each potential state has to be considered 
(e.g., for DNA the states 'A' and 'G' are the potential states that are 
considered when the observed state is 'R'). The scaled-likelihood is then~: 
\begin{equation}
L^{*}(s) =
\sum_{g}^{G}\sum_{h \in \mathcal{A}}\sum_{h^{\prime} \in \mathcal{A}}
p_{g}
\pi_h
L^{*}(s=h|V)
P_{hh^{\prime}}(l \times r_g)
L^{*}(s=h^{\prime}|U)
\end{equation}

The (unscaled) log likelihood is then~: 
\begin{equation}
ln(L(s)) = ln(L^{*}(s)) + ln(Z_s(U)) + ln (Z_s(V))
\end{equation}


Let $\nu$ be the expected frequency of invariable sites or invariants. 
Invariant are peculiar sites that do not sustain any mutation. Let
$L(s|inv=1)$ be the probability of site $s$ given that it is 
an invariant. $L(s|inv=0)$ is the probability of site $s$ given that
it is not an invariant. We have~:
\begin{equation}\label{eq:loglikelihood}
\nonumber ln(L(s)) = ln(L(s|inv=0)\times(1 - \nu) + L(s|inv=1)\times \nu)
\end{equation}

$L(s|inv=0)$ is the logarithm of the right-hand side of equation \ref{eq:loglikelihood}. If no polymorphism is observed at site $s$ and that the state at this site is $h$, the log likelihood is then~: 
\begin{equation}\label{eq:invariant1}
ln(L(s)) = ln(L(s|inv=0)\times(1 - \nu) + \pi_h \times \nu)
\end{equation}

If polymorphism is observed, the site can not be an invariant. The log likelihood is therefore~: 
\begin{equation}\label{eq:invariant2}
ln(L(s)) = ln(L(s|inv=0)\times(1 - \nu)
\end{equation}


\end{document}


%%%Emacs : Local Variables:
%%%Emacs : 8-bit-save-8-bit: t
%%%Emacs : 8-bit-TeX-convention: frenchTeX
%%%Emacs : mode : 8-bit
%%%Emacs : End :
